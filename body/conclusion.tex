%%==================================================
%%conclusion.tex for SJTU Course Design Thesis
%% based on CASthesis, SJTU master thesis
%% modified by icetiny@gmail.com
%% version: 0.3a
%% Encoding: UTF-8
%% last update: Dec 5th, 2010
%%==================================================

\chapter*{全文总结\markboth{全文总结}{}}
\addcontentsline{toc}{chapter}{全文总结}
本文对本次课程设计工作作了总结。首先介绍了硬件设计部分设计。有了硬件设计的基础之后,功能实现成为可能。第\ref{chap:functionintro}章阐述了信号灯系统具体的功能,并对各种调节时间参数的方式和键盘操作了详细介绍。第\ref{chap:software}章则主要介绍了软件程序的结构和设计思路,对一些关键问题如定时参数、中断优先级等作了讨论。

总体来看,我们设计的系统达到了预先设定的要求,也完成了我们自己定下的扩展要求。仿真效果良好,反应迅速,定时精确,操作便捷。但是我们也认识到,将这个系统投入实际应用,硬件调试仍是必须的。


\textbf{待增加的功能} 

以下这些功能主要也是软件实现的问题,难度并不大,但是由于时间紧迫,目前的程序并未包含,今后如有机会继续改进,可以考虑增加这些功能。
\begin{enumerate}
\item 通过键盘切换入紧急状态下,此状态下所有路口均显示红灯,倒计时不工作。
\item 通过键盘切换入夜间模式,此状态下所有路口均有闪烁的黄灯。
\item 加入黄灯闪烁的功能。
\item \ldots
\end{enumerate}