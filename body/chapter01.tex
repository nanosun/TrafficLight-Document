%%==================================================
%% chapter01.tex for SJTU Course Design Thesis
%% based on CASthesis ,SJTU Master Thesis
%% modified by icetiny@gmail.com
%% version: 0.3a
%% Encoding: UTF-8
%% last update: Dec 5th, 2010
%%==================================================

%\bibliographystyle{sjtu2} %[此处用于每章都生产参考文献]
\chapter{设计背景及任务}
\label{chap:intro}
\section{设计背景}
交通信号灯是城市交通行为中重要的环节,它带来有秩序的交通,引导人流车流、缓解拥堵、提高通行效率。

传统的交通信号灯简单地实现红黄绿等的切换,时间间隔难以调节,高峰时间和夜间红绿灯的间隔不变,更不可能根据不同路口的车流量做出调节。单调不变的红灯会加剧人的焦急感,人们要求信号灯能够有显示倒计时的功能。另外,对于闯红灯、超速等交通违规现象的监控,如需要整合到路口信号灯系统中。

要控制具有如上所有功能的交通信号灯系统,单片机因其运行速度快、功耗低、成本低廉、可编程性优越,接口丰富等优点成为控制核心的必然选择。
\section{设计任务}
结合设计背景,要求设计一个以S52单片机为核心的交通信号灯控制系统,同时对闯红灯车辆进行拍照。
\subsection{基本设计要求}
根据交通规则设计
\begin{enumerate}
\item 十字路口交通信号灯分为红灯、绿灯和黄灯,交替亮灭,保证车辆安全有序通行;
\item 亮灯时间可以设置;
\item 路口安装摄像监控装置,对于闯红灯车辆进行拍摄,存入录像机;
\item 操作简单。
\end{enumerate}
\subsection{扩展设计要求}
在基本设计要求的基础上,结合日常的生活经验、以及我们对单片机技术的掌握,我们认为最终设计的系统还具有以下功能:
\begin{enumerate}
\item 可以实时显示两组红绿灯倒计时时间;
\item 相交的两条路应有不同的红绿灯时间;
\item 亮灯时间可以根据当前所处的时段自动调整;
\item 亮灯时间可以通过监控车流量等信息的上位机智能调节;
\item 同一路口的红绿灯之间有互锁控制;
\item 设置界面应友好易用。
\end{enumerate}


