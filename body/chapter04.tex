%%==================================================
%% chapter04.tex for SJTU Course Design Thesis
%% based on CASthesis, SJTU master thesis
%% modified by icetiny@gmail.com
%% version: 0.3a
%% Encoding: UTF-8
%% last update: Dec 5th, 2010
%%==================================================

% \bibliographystyle{sjtu2} %[此处用于每章都生产参考文献]

\chapter{软件设计及问题研讨}
\label{chap:software}
本章是本篇论文的重点,因为本人在本次课程设计工作中主要做的的就是程序设计,对于这一部分有许多的了解最为深刻。首先我将阐述程序设计的基本思路,对重要的变量和子程序做解释。另外还将就程序设计中一些比较关键的问题做分析。

\section{程序概览}
本程序用全部用汇编语言编写,整体大约六百四十余行,编译后约占1.3Kb。附录\ref{app:code}提供了完整的源代码。
\subsection{红灯时间参数的存储和调用}
	本程序中最重要的参数就是两个路口的两个红灯时间参数,倒计时、切换灯的状态、设置时间参数等都是围绕这两个参数进行的。要解释清楚该程序,首先就需要解释这个参数的不同形式,和它们的存储和调用方式。
	表\ref{tab:timeregister}列出了相关的参数作用和存储方式。在这些存储空间中,数据都是以BCD码的形式存储的,这是为了用户输入和显示的方便\footnote{但是本程序中标志时区的量都是以十六进制的形式存储的,这是因为时区参数与查表有关,而表格时连续的,16进制的时区参数对于查表是便捷的。}。
	
	这个表中数据是从底部行流向顶部行描述的存储空间的。而最后两行展示了ROM表与RAM表之间的优先级关系。
	\begin{table}[!htpb]
      	\centering
      	\bicaption[tab:timeregister]{时间参数表}{时间参数表}{Table}{Time registers}
      	\begin{tabular}{p{0.2\textwidth}|p{0.7\textwidth}} \toprule
        \multicolumn{1}{c|}{存储形式} & \multicolumn{1}{c}{参数功能}   \\ \midrule   
				\multicolumn{1}{c|}{R2\quad R3} & 分别存储当前两个倒计时数码管组显示的内容,R2为红灯,R3为绿灯(或黄灯) \\ \cmidrule(lr){1-2}
				\multicolumn{1}{c|}{30H\quad 31H} & 存储当前时段的两个红灯时间参数,倒计时到0后,R2和R3会从这两个字节重新载入倒计时时间。三种情况改变这两个字节的值:
				\begin{itemize} \item 当一个小时结束时; \item 用户通过键盘定义了当前时段的时间参数时,立即刷新,立即生效;\item 上位机发送包要求改变这两个位时。\end{itemize} \\ \cmidrule(lr){1-2}
				ROM中的表\#TAB\_LIGHT \_TIME & 是预置在ROM中的时间参数表,按小时排列,每小时两个,共48个字节。当一小时计数到时,30H和31H从这个表中载入新的数据。\\ \cmidrule(lr){1-2}
				RAM中50H到7FH & 这48个字节存储了用户通过键盘定义的时间参数数据。它的优先级比ROM中的表要高。但如果这个表中的数据为0。则视为未定义,程序仍从ROM表中读取数据存入30H和31H。 \\ 				 \bottomrule
      	\end{tabular}
		\end{table}
\subsection{中断的应用}
\begin{table}[!htpb]
      	\centering
   	\bicaption[tab:intvectors]{中断应用表}{中断应用表}{Table}{Interrupt vectors}
      	\begin{tabular}{ccc|p{0.45\textwidth}} \toprule
        中断优先级 & 中断名 & 中断处理程序 & 中断功能\\ \midrule
        1 & 计时器0 & INT\_T0 & 计时$\frac{1}{144}$ 秒,清零看门狗,取反计数器2的输入端\\ \hline
        2 & 计数器1 & INT\_C1 & 计数36,即0.5秒,中断处理程序完成倒计时等工作\\ \hline
        3 & 外部中断0 & INT\_EX0 
					& 响应地感线圈输入信号 \\ \hline
				4 & 外部中断1 & INT\_EX1 
					& 响应按键输入信号\\	\bottomrule
      	\end{tabular}
		\end{table}
为了提高程序响应的速度,减少CPU的工作量,所有功能都移入中断处理程序中,主函数只起到初始化的作用。表\ref{tab:intvectors}列出了中断的应用方式。中断优先级的设置将在\ref{sec:intstudy}中讨论。此外,若要实现双机多机通讯,串口中断自然也需要应用。
\subsection{子程序一览}   	   	           	\begin{longtable}[!htbp]{p{0.05\textwidth}|p{0.13\textwidth}|p{0.3\textwidth}|p{0.2\textwidth}|p{0.2\textwidth}}\toprule \endhead \hline \endfoot
	   	  
        序号 & 子程序名 & 主要功能 & 主要参量& 包含子程序 \\ \hline 
        1 & INT\_T0 & 清零看门狗,取反计数器2的输入端 & &\\ \hline
        2 & INT\_C1 & 倒计时减一显示,切换灯状态,计时,重载时间参数 & R0灯状态,R6秒计数,R7分计数 &  SUBBCD, CLOSEDIG, DISPLAY\_NUMBER, CHANGELIGHT, GET\_TIME\_LIGHT等\\ \hline
        3 & INT\_EX0 
					& 响应地感线圈输入信号,向摄像头输出开关量控制信号\tnote{1} & & DELAY10\\ \hline
				4 & INT\_EX1 
					& 响应按键输入信号,完成键盘操作流程所要求的功能。& R4存储当前屏幕状态(显示内容的标志),3AH当前选择的时间段,3BH路口A的正在被设定的数据,3CH相应的路口B的数据 位7F标志当前调整的是高位还是低位& DELAY10, DELAY4MS5, HEX2BCD, DISPLAY\_NUMBER, GET\_TIME\_LIGHT, BCDINC, BCDDEC\\ \hline
				5 & BCDINC &根据选位标志7F对高位或低位做加一运算,对9加一得到0 & 3DH传递被处理的数据&\\ \hline
				6 & BCDDEC &根据选位标志7F对高位或低位做减一运算,对0减一得到9 & 3DH传递被处理的数据&\\ \hline		
				7 & CLOSEDIG &根据选位标志,关闭某一位的显示& 3EH传递被处理的数据&通过调用DISPLAY \_NUMBER送显示\\ \hline 
				8 & HEX2BCD &将16进制数转为BCD码,主要用于时区的显示 & 38H传递被处理的数据&\\ \hline		
				9 & GET TIME LIGHT &根据当前的时区,刷新30H和31H中的数据 & 3DH传递被处理的数据&\\ \hline		
				10 & CHANGE LIGHT &根据目前的灯状态,选通灯锁存器,点亮相应灯 & R0指向存储亮灯数值的单元(32H\-31H)&\\ \hline		
				11 & DISPLAY NUMBER &将BCD码显示到数码管上,一次显示相邻的两位& 38H存放BCD码,39H存放低位数码管的地址& GETDIGIT, DISDIGIT\\ \hline	
				12 & GETDIGIT &查表取数码管段码 & 取得段码存入38H&\\ \hline	
				13 & DISDIGIT &将38H中的段码送入数码管显示 & &\\ \hline			
				14 & SUBBCD &BCD码减一 & 对36H中的BCD码操作&\\ \hline	
      	\end{longtable}
\section{问题研讨及程序分析}
\subsection{定时程序的设计} \label{sec:timeprog}
本程序中利用T0和T1串联产生0.5秒的定时。T0定时到后,取反P3.5位,即送给T1一个二分频的计数信号。主程序放在T1中断处理程序中执行,提高了响应速度和计时准确性。
理论上T0的定时时间应越长越好,这样与T1同时触发中断的概率越小,但是考虑到S52单片机看门狗需要每隔8192个机器周期就会产生复位信号,而T0的中断处理程序中清零看门狗,故T0的计数周期必须小于8192个机器周期。实际设定T0每6400个机器周期产生中断,计数器T1的计数个数为36。

若按照以上设定,则T0的时间常数为\#E700H。但是考虑到本程序要在一天中累计小时数,计时精度要求较高。仿真测试显示该时间常数还需要修正。经估算和仿真测试发现,T0的中断处理程序约有9个周期的滞后,故修正时间常数为\#E709H。

计数器1的中断处理程序长度远小于6400×2个机器周期,故与计时器0的中断不会发生冲突。

\begin{lstlisting}[language={[x86masm]assembler},escapeinside="",caption={时间常数设置},label=code:timeconstantset] 
		MOV TMOD,#61H	;初始化计时器 定时器0方式1 计数器1方式2
		MOV TH0,#0E7H	
		MOV TL0,#09H	
		MOV TH1,#0DCH ;256-36
		MOV TL1,#0DCH
\end{lstlisting} 
经仿真测试,按照如代码\ref{code:timeconstantset}设置后,运行时间如表\ref{tab:timeset}所示,每秒偏差小于$5\mu s$,每天的偏差约为0.4s还是比较精确的。代码中T0中断中还需先插入一nop语句。
\begin{table}[!htpb]
      	\centering
   	\bicaption[tab:timeset]{时间偏差}{定时偏差}{Table}{Time deviation}
      	\begin{tabular}{ccccc} \toprule
       		\multicolumn{5}{c}{每次T1中断时的时刻(ms)} \\ \hline
       		493.44& 993.44&1493.43&1993.43&2493.43 \\ 
       		2993.43&3493.43&3993.42&4492.42&4492.42 \\ \bottomrule
      	\end{tabular}
		\end{table}
\subsection{中断优先级的设计} \label{sec:intstudy}
\begin{lstlisting}[language={[x86masm]assembler},escapeinside="",caption={中断优先级设置}] 
		MOV IP,#00001010B  
		; \verb+EA | - | PT2 | PS | PT1 | PX1 | PT0 | PX0+
		; \verb+1  | 0 |  0  |  0 |  1  |  0  |  1  |  0+
\end{lstlisting} 
表\ref{tab:intvectors}已经给出了中断优先级的排列顺序。T0中断的优先级最高,因为它是精确定时的关键。T1计数器次之,设为高优先级是防止其被持续时间较长的按键中断阻断。T1中断程序关乎倒计时的正确显示,而键盘中断可能长达数百毫秒,若将T1阻断,则倒计时也将延迟数百毫秒,这是不能接受的。两个外部中断都处于低优先级,除了它们耗时较长不能阻断计时中断的原因,它们本身也不需要非常迅速的响应,用户的的反应在百毫秒级别,闯红灯拍照计时延迟一秒之久,也仍能拍到肇事车辆。
\subsection{看门狗程序}
根据文献\cite{mcu1},设计看门狗程序如代码所示。S52单片机内置看门狗硬件,它的作用是防止程序跑飞。但也存在一个问题:复位后用户定义的数据将被清空。
\begin{lstlisting}[language={[x86masm]assembler},escapeinside="",caption={看门狗程序}] 
		;主程序中喂狗
		MOV 0A6H,#01EH ;激活看门狗
		MOV 0A6H,#0E1H
		
		;INT\_T0 中断处理程序中清零看门狗
		INT_TO:								;计时器0中断处理程序
		nop
		MOV TH0,#0E7H 				
		MOV TL0,#09H
		MOV 0A6H,#01EH 				;清零看门狗
		MOV 0A6H,#0E1H 
		CPL P3.5
		RETI
\end{lstlisting} 
\subsection{双机通讯}
双机通讯由于难以仿真,故没有放到最终的源程序中。但我们考虑到了这个功能,并设计了它的接口程序。
\begin{lstlisting}[language={[x86masm]assembler},escapeinside="",caption={双机通训代码}] 
		;主机从机设置相同的波特率,为$\frac{1}{64}f_{osc}$
		MOV PCON #80H
		MOV SCON #80H
		;从机还需要设置 SETB SM2
		;从机中断处理程序
		CLR RI
		如果SBUF是#FFH,则CLR SM2 开始接收数据
		接受一帧地址a
		接受一帧红灯时间,直接存入a
		如果SBUF是#FEH,则SETB SM2 结束接受数据
		;主机发送包程序
		SETB TB8
		发送 #0FFH
		CLR TB8
		发送一帧地址
		发送一帧数据
		发送#0FEH 结束发送
\end{lstlisting}
\subsection{按键去抖}
在按键中断处理程序中,调用DELAY10延时10ms进行前沿去抖。考虑到键盘处理程序较长,且最后总要送数码管显示,所以去后沿抖动时仅延时4.5ms。而且去抖是通过循环判断是否已撤键,避免对同一次按键的多次响应。这也要求延时程序不能耗时太长,避免过长的延迟。
\subsection{BCD码的处理}
在本程序中BCD码的应用很多。

通过HEX2BCD程序将16进制数转换为BCD码,主要思路是除以10。

BCD码的加一可通过以下命令简单实现,汇编中的DA命令必须紧跟在ADD或ADDC命令之后。 
\begin{lstlisting}[language={[x86masm]assembler},escapeinside="",caption={BCD码加一,结果存入ACC}] 
		MOV A, BCD码
		ADD A,#1
		DA A
\end{lstlisting}

但是汇编中对减一没有相应的命令,本程序中有子程序SUBBC可以实现此功能。主要思路是如果出现借位,个位会出现F,则再减6将个位变为9即可。
\begin{lstlisting}[language={[x86masm]assembler},escapeinside="",caption={SUBBCD程序,BCD码减一}] 
		SUBBCD:						;BCD码减一,对36H中的数做BCD码减1
		PUSH ACC
		PUSH PSW
		DEC 36H
		MOV A,36H
		ANL A,#0FH
		CJNE A,#0FH,SUBBCD_EXIT
		CLR CY
		MOV A,36H
		SUBB A,#06
		MOV 36H,A
SUBBCD_EXIT:
		POP PSW
		POP ACC
		RET
\end{lstlisting}

本程序中另外写有子程序BCDINC和BCDDEC,它们是为了配合键盘操作中对屏显数字做加减操作而设计的。除了实现个位和十位的加减,它还考虑到了0-9之间循环的问题。由于键盘操作的要求比较复杂,故直接编写了这两个新子程序,而没有重用SUBBCD程序。而且SUBBCD程序调用的频率很高,比键盘操作的频率高的多,所以程序不宜写的太臃肿,运行效率是考虑的一个重要因素。