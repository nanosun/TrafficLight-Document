%%==================================================
%% abstract.tex for SJTU project design thesis
%% based on CASthesis
%% modified by icetiny@gmail.com
%% version: 
%% Encoding: UTF-8
%% last update: Dec 5th, 2010
%%==================================================

\begin{abstract}
	机电控制技术在生产生活中发挥着巨大的作用,尤其是以单片机为核心器件的控制系统因其功耗低、可编程性强、扩展能力强等优点应用日益广泛。交通信号灯是社会维持正常交通秩序的重要工具。信号灯对交通参与者的友好性及其自身的可编程性,在机动车数量迅猛增长的当下凸显其重要性。在机电控制技术课程学习之后,我们尝试利用单片机技术及相关软硬件技术架设一交通信号灯控制系统,整合倒计时、对闯红灯者拍照等功能,并实现红绿灯持续时间的简便可调。本文将对该系统的设计和功能使用做出说明,并讨论一些可能的拓展功能。
	
	由于本人在课程设计过程中主要负责汇编程序的编写,本文也将更为侧重软件部分:程序设计的思路、主要代码的解释、关键问题的探讨等。同时也会较对设计过程中软件设计和硬件设计之间的相互影响做一些阐述。
  

  \keywords{\large 单片机 \quad 控制 \quad 汇编}
\end{abstract}
\label{chap:abstract}

